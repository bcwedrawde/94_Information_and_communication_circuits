\documentclass[12pt]{article}
\usepackage{pmmeta}
\pmcanonicalname{LinearCode}
\pmcreated{2013-03-22 14:21:24}
\pmmodified{2013-03-22 14:21:24}
\pmowner{mathcam}{2727}
\pmmodifier{mathcam}{2727}
\pmtitle{linear code}
\pmrecord{7}{35836}
\pmprivacy{1}
\pmauthor{mathcam}{2727}
\pmtype{Definition}
\pmcomment{trigger rebuild}
\pmclassification{msc}{94B05}
\pmrelated{CyclicCode}
\pmrelated{WeightEnumerator}
\pmrelated{DualCode}
\pmrelated{EvenCode}
\pmrelated{AutomorphismGroupLinearCode}
\pmdefines{binary code}
\pmdefines{ternary code}
\pmdefines{quaternary code}
\pmdefines{dimension of a linear code}

% this is the default PlanetMath preamble.  as your knowledge
% of TeX increases, you will probably want to edit this, but
% it should be fine as is for beginners.

% almost certainly you want these
\usepackage{amssymb}
\usepackage{amsmath}
\usepackage{amsfonts}
\usepackage{amsthm}

% used for TeXing text within eps files
%\usepackage{psfrag}
% need this for including graphics (\includegraphics)
%\usepackage{graphicx}
% for neatly defining theorems and propositions
%\usepackage{amsthm}
% making logically defined graphics
%%%\usepackage{xypic}

% there are many more packages, add them here as you need them

% define commands here

\newcommand{\mc}{\mathcal}
\newcommand{\mb}{\mathbb}
\newcommand{\mf}{\mathfrak}
\newcommand{\ol}{\overline}
\newcommand{\ra}{\rightarrow}
\newcommand{\la}{\leftarrow}
\newcommand{\La}{\Leftarrow}
\newcommand{\Ra}{\Rightarrow}
\newcommand{\nor}{\vartriangleleft}
\newcommand{\Gal}{\text{Gal}}
\newcommand{\GL}{\text{GL}}
\newcommand{\Z}{\mb{Z}}
\newcommand{\R}{\mb{R}}
\newcommand{\Q}{\mb{Q}}
\newcommand{\C}{\mb{C}}
\newcommand{\<}{\langle}
\renewcommand{\>}{\rangle}
\begin{document}
Often in coding \PMlinkescapetext{theory}, a code's alphabet is taken to be a finite field.  In particular, if $A$ is the finite field with two (resp. three, four, etc.) elements, we call $C$ a binary (resp. ternary, quaternary, etc.) code.  In particular, when our alphabet is a finite field then the set $A^n$ is a vector space over $A$, and we define a \emph{linear code over $A$} of block length $n$ to be a subspace (as opposed to merely a subset) of $A^n$.  We define the \emph{dimension of $C$} to be its dimension as a vector space over $A$.

Though not sufficient for unique classification, a linear code's block length, dimension, and minimum distance are three crucial parameters in determining the strength of the code.  For referencing, a linear code with block length $n$, dimension $k$, and minimum distance $d$ is referred to as an $(n,k,d)$-code.

Some examples of linear codes are Hamming Codes, BCH codes, Goppa codes, Reed-Solomon codes, and the \PMlinkname{Golay code}{BinaryGolayCode}.
%%%%%
%%%%%
\end{document}
