\documentclass[12pt]{article}
\usepackage{pmmeta}
\pmcanonicalname{DerivationOfHartleyFunction}
\pmcreated{2013-03-22 14:32:15}
\pmmodified{2013-03-22 14:32:15}
\pmowner{Mathprof}{13753}
\pmmodifier{Mathprof}{13753}
\pmtitle{derivation of Hartley function}
\pmrecord{11}{36082}
\pmprivacy{1}
\pmauthor{Mathprof}{13753}
\pmtype{Derivation}
\pmcomment{trigger rebuild}
\pmclassification{msc}{94A17}

\endmetadata

% this is the default PlanetMath preamble.  as your knowledge
% of TeX increases, you will probably want to edit this, but
% it should be fine as is for beginners.

% almost certainly you want these
\usepackage{amssymb}
\usepackage{amsmath}
\usepackage{amsfonts}

% used for TeXing text within eps files
%\usepackage{psfrag}
% need this for including graphics (\includegraphics)
%\usepackage{graphicx}
% for neatly defining theorems and propositions
%\usepackage{amsthm}
% making logically defined graphics
%%%\usepackage{xypic}

% there are many more packages, add them here as you need them

% define commands here
\begin{document}
\PMlinkescapeword{property}
\PMlinkescapeword{satisfy}

We want to show that the Hartley function $\log_2(n)$ is the only
function mapping natural numbers to real numbers that
\PMlinkescapetext{satisfies}
\begin{enumerate}
\item $H(mn) = H(m)+H(n)$ (\PMlinkescapetext{additivity}),

\item $H(m) \leq H(m+1)$ (monotonicity), and

\item $H(2)=1$  (normalization).
\end{enumerate}

Let $f$ be a function on positive integers that satisfies the above three properties. Using the additive property, it is easy to see that the value of $f(1)$ must be zero. So we want to show that $f(n) = \log_2(n)$ for all integers $n\geq 2$.


From the additive property, we can show that for any integer $n$
and $k$,
\begin{equation}
f(n^k) = kf(n).
\end{equation}


Let $a>2$  be an integer.  Let $t$ be any positive integer. There is a unique
integer $s$ determined by
\[
  a^s \leq 2^t < a^{s+1}.
\]
Therefore,
\[
  s \log_2 a\leq t < (s+1) \log_2 a
\]
and
\[
 \frac{s}{t} \leq \frac{1}{\log_2 a} < \frac{s+1}{t}.
\]

On the other hand, by monotonicity,
\[
 f(a^s) \leq f(2^t) \leq f(a^{s+1}).
\]
Using Equation (1) and $f(2)=1$, we get
\[
 s f(a) \leq t  \leq (s+1) f(a),
\]
and
\[
 \frac{s}{t} \leq \frac{1}{f(a)} \leq \frac{s+1}{t}.
\]

Hence,
\[
 \Big| \frac{1}{f(a)} - \frac{1}{\log_2(a)} \Big| \leq
 \frac{1}{t}.
\]

Since $t$ can be arbitrarily large, the difference on the left
hand \PMlinkescapetext{side} of the above inequality must be zero,
\[
 f(a) = \log_2(a).
\]

%%%%%
%%%%%
\end{document}
