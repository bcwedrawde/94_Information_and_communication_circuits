\documentclass[12pt]{article}
\usepackage{pmmeta}
\pmcanonicalname{DualCode}
\pmcreated{2013-03-22 15:13:29}
\pmmodified{2013-03-22 15:13:29}
\pmowner{GrafZahl}{9234}
\pmmodifier{GrafZahl}{9234}
\pmtitle{dual code}
\pmrecord{6}{36989}
\pmprivacy{1}
\pmauthor{GrafZahl}{9234}
\pmtype{Definition}
\pmcomment{trigger rebuild}
\pmclassification{msc}{94B05}
%\pmkeywords{linear code}
%\pmkeywords{orthogonal complement}
\pmrelated{LinearCode}
\pmrelated{OrthogonalComplement}
\pmdefines{self-dual}
\pmdefines{self-orthogonal}

% this is the default PlanetMath preamble.  as your knowledge
% of TeX increases, you will probably want to edit this, but
% it should be fine as is for beginners.

% almost certainly you want these
\usepackage{amssymb}
\usepackage{amsmath}
\usepackage{amsfonts}

% used for TeXing text within eps files
%\usepackage{psfrag}
% need this for including graphics (\includegraphics)
%\usepackage{graphicx}
% for neatly defining theorems and propositions
%\usepackage{amsthm}
% making logically defined graphics
%%%\usepackage{xypic}

% there are many more packages, add them here as you need them

% define commands here
\newcommand{\Prod}{\prod\limits}
\newcommand{\Sum}{\sum\limits}
\newcommand{\mbb}{\mathbb}
\newcommand{\mbf}{\mathbf}
\newcommand{\mc}{\mathcal}
\newcommand{\ol}{\overline}

% Math Operators/functions
\DeclareMathOperator{\Frob}{Frob}
\DeclareMathOperator{\cwe}{cwe}
\DeclareMathOperator{\we}{we}
\DeclareMathOperator{\wt}{wt}
\begin{document}
\PMlinkescapeword{self-dual}
Let $C$ be a linear code of block length $n$ over the finite field
$\mbb{F}_q$. Then the set
\begin{equation*}
C^\perp:=\{d\in\mbb{F}_q^n\mid c\cdot d=0\text{ for all }c\in C\}
\end{equation*}
is the \emph{dual code} of $C$. Here, $c\cdot d$ denotes either the
standard dot product or the Hermitian dot product.

This definition is reminiscent of orthogonal complements of \PMlinkid{finite
dimensional}{5398} vector spaces over the real or complex numbers. Indeed,
$C^\perp$ is also a linear code and it is true that if $k$ is the
\PMlinkid{dimension}{5398} of $C$, then the \PMlinkescapetext{dimension} of
$C^\perp$ is $n-k$. It is, however, \textbf{not} necessarily true that
$C\cap C^\perp=\{0\}$. For example, if $C$ is the binary code of block
length $2$ \PMlinkid{spanned by}{806} the codeword $(1,1)$ then $(1,1)\cdot(1,1)=0$,
that is, $(1,1)\in C^\perp$. In fact, $C$ equals $C^\perp$ in this
case. In general, if $C=C^\perp$, $C$ is called
\emph{self-dual}. Furthermore $C$ is called \emph{self-orthogonal} if
$C\subseteq C^\perp$.

Famous examples of self-dual codes are the extended binary Hamming
code of block length $8$ and the extended binary Golay code of block
length $24$.
%%%%%
%%%%%
\end{document}
