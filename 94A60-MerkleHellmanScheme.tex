\documentclass[12pt]{article}
\usepackage{pmmeta}
\pmcanonicalname{MerkleHellmanScheme}
\pmcreated{2013-03-22 15:12:40}
\pmmodified{2013-03-22 15:12:40}
\pmowner{aoh45}{5079}
\pmmodifier{aoh45}{5079}
\pmtitle{Merkle-Hellman scheme}
\pmrecord{5}{36972}
\pmprivacy{1}
\pmauthor{aoh45}{5079}
\pmtype{Definition}
\pmcomment{trigger rebuild}
\pmclassification{msc}{94A60}
\pmsynonym{Merkle-Hellman cryptosystem}{MerkleHellmanScheme}

\usepackage{amssymb}
\usepackage{amsmath}
\usepackage{amsfonts}
\usepackage{amsthm}

% need this for including graphics (\includegraphics)
%\usepackage{graphicx}
% making logically defined graphics
%%%\usepackage{xypic}
\begin{document}
\PMlinkescapeword{generates}
\PMlinkescapeword{arithmetic}
\PMlinkescapeword{length}

The \emph{Merkle-Hellman} cryptosystem was one of the earliest examples of public key cryptography, and depends on the NP-complete problem ``SUBSET SUM'' for its security. 

Suppose Bob wants to send Alice a message. 

Alice generates private key ${a_1, a_2, \dots a_n}$ which is a superincreasing sequence. She then picks $d \gg \sum_{i=1}^n a_i$ and $h$ comprime with $d$. Using the euclidean algorithm, she finds $h^{-1}$ such that $hh^{-1} \equiv 1 \quad (\textrm{mod }d)$.

Alice now generates her public key ${b_1, b_2, \dots b_n}$ where $b_i = ha_i\quad (\textrm{mod }d)$ and sends this to Bob.

Bob breaks up his message into binary strings of length $n$. To send the string $m_1 m_2 \dots m_n$ to Alice, he forms $C = \sum_{i=1}^n m_i b_i$ and sends $C$ to Alice.

On receiving $C$, Alice forms $V = h^{-1}C \quad (\textrm{mod }d)$. Now, since $b_i = ha_i\quad (\textrm{mod }d)$, we have that $V = \sum_{i=1}^n m_i a_i$. Since ${a_i}$ is a superincreasing sequence, it is easy to recover the $m_i$ if you know $V$ and ${a_i}$, and it takes $O(n)$ arithmetic operations.

In 1982, a fast algorithm was found for recovering the message knowing only the public key and the cryptogram $C$. It takes advantage of the fact that the public key ${b_i}$ is not generated in a random way, but comes from a superincreasing sequence.
%%%%%
%%%%%
\end{document}
