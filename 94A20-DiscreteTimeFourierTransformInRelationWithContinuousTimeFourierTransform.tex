\documentclass[12pt]{article}
\usepackage{pmmeta}
\pmcanonicalname{DiscreteTimeFourierTransformInRelationWithContinuousTimeFourierTransform}
\pmcreated{2013-03-22 17:37:45}
\pmmodified{2013-03-22 17:37:45}
\pmowner{fernsanz}{8869}
\pmmodifier{fernsanz}{8869}
\pmtitle{discrete time Fourier transform in relation with continuous time Fourier transform}
\pmrecord{10}{40050}
\pmprivacy{1}
\pmauthor{fernsanz}{8869}
\pmtype{Theorem}
\pmcomment{trigger rebuild}
\pmclassification{msc}{94A20}
%\pmkeywords{Sampling}
%\pmkeywords{Fourier Transform}
%\pmkeywords{Discrete Time Fourier Transform}
%\pmkeywords{DTFT}
%\pmkeywords{replicas}
\pmrelated{SamplingTheorem}
\pmrelated{ApproximatingFourierIntegralsWithDiscreteFourierTransforms}
\pmrelated{Distribution4}
\pmrelated{SpaceOfRapidlyDecreasingFunctions}
\pmdefines{DTFT}
\pmdefines{Discrete Time Fourier Transform}

% this is the default PlanetMath preamble.  as your knowledge
% of TeX increases, you will probably want to edit this, but
% it should be fine as is for beginners.

% almost certainly you want these
\usepackage{amssymb}
\usepackage{amsmath}
\usepackage{amsfonts}

% need this for including graphics (\includegraphics)
\usepackage{graphicx}
% for neatly defining theorems and propositions
\usepackage{amsthm}

% there are many more packages, add them here as you need them

% define commands here

% THEOREMS -------------------------------------------------------
\newtheorem{thm}{Theorem}
\newtheorem{cor}{Corollary}
\newtheorem{lem}{Lemma}
\newtheorem{prop}{Proposition}
\theoremstyle{definition}
\newtheorem{defn}{Definition}
\theoremstyle{remark}
\newtheorem{rem}{Remark}
\newtheorem*{notat}{Notation}
%\numberwithin{equation}{section}
% MATH -----------------------------------------------------------
\newcommand{\abs}[1]{\left\vert#1\right\vert}
\newcommand{\R}{\mathbb R}
\newcommand{\C}{\mathbb C}
\newcommand{\Z}{\mathbb Z}
\newcommand{\To}{\rightarrow}
\newcommand{\xd}{x_d[n]}
\newcommand{\xc}{x(t)}
\newcommand{\D}{\mathcal{D}}
\newcommand{\rapid}{\mathcal{S}}
\newcommand{\replica}[3]{\sum_{k=#2}^{#3} #1 \left(\frac{\omega- 2\pi \cdot k}{T}
\right)}
\newcommand{\DTFT}[1]{\sum_{k=-\infty}^{+\infty} \frac{#1(kT^+)+#1(kT^-)}{2} e^{-j\omega
k}}

\begin{document}
\title{Fourier Transforms of Sampled Signals}%
\author{Fernando Diego Sanz Gamiz}%

% ----------------------------------------------------------------
\textbf{Introduction}

Computers are able to handle only finite number of data. Hence, if
we are to study and treat real world signals (i.e. functions $\R \To
\C$) in a computer, a way to characterize signals by a finite number
of data has to be found.


\noindent If we sample the values of the signal $\xc$ at periodic
times $t=nT, n \in \Z$ we form the sequence $\xd=x(nT)$. Does this
sequence contain all the information relative to $\xc$?


\noindent We know from the sampling theorem that if the signal $\xc$
is bandlimited, the sampled sequence allows us to recover the
original continuous time signal provided the sampling frequency
$1/T$ is at least twice the maximum frequency of the signal. However,
real signals are not of finite bandwidth as this would imply the
signal to be of infinite time duration. Therefore, a problem arise
of how well can we approximate the original signal $\xc$ by the
sampled sequence $x_d[n]$. In fact, we are interested in studying
the spectrum of the original signal based upon the samples $\xd$.


\noindent While the relation between $\xd$ and the spectrum of $\xc$
is widely used in communication and electronic engineering books, it
is difficult to find a rigorous proof. We cover here the gap between
engineering daily knowledge and rigorous mathematical proof of the
named relations establishing under what assumptions those relations
are valid.


\textbf{Relation between Discrete Time Fourier Transform and Continuous time Fourier transform}

\begin{thm}
Let $x:\R \To \C$ be a bounded variation (it can be, in particular,
piecewise smooth) $L^1$ function and let $X(\omega)$ be its Fourier
transform \footnote{in the present entry we will take the Fourier
transform of $x(t)$ to be $X(\omega)=\int_{-\infty}^{+\infty}
x(t)e^{-jwt} d\omega$}. Let $T \in \R$. If $g_n(\omega)
=\replica{X}{-n}{+n}$ converges a.e as $n \To +\infty$ and is there
is $M$ such that $\abs{g_n(\omega)}<M$ a.e. then \label{eq1}
\begin{equation}
\frac {1}{T} \replica{X}{-\infty}{+\infty} = \DTFT{x} \mbox{\,\,\,
in } L^2([-\pi, \pi])
\end{equation}

If additionally $\replica{X}{-\infty}{+\infty}$ is continuous, the
right hand side of (\ref{eq1}) converges uniformly to the left hand
side in any closed interval $[a,b] \subset [-\pi, \pi]$.
\end{thm}

\begin{proof}
By hypothesis we can form the function $$g(\omega)=\lim_{n \To
+\infty } g_n(\omega)= \lim_{n \To +\infty } \replica{X}{-n}{+n}$$
This function is obviously periodic of period $2 \pi$ and bounded,
hence it can be expanded in its Fourier series which converge in
$L^2([-\pi, \pi])$; the Fourier theory shows that the convergence is
uniformly in $[a,b] \subset [-\pi, \pi]$ if $g(\omega)$ is
continuous.
$$g(\omega)= \replica{X}{-\infty}{+\infty} =
\sum_{k=-\infty}^{+\infty} c_ke^{j\omega k}$$ where the coefficients
$c_k$ are given by $$ c_k = \frac{1}{2\pi} \int_{-\pi}^{\pi}
g(\omega) e^{-j\omega k} d\omega = \frac{1}{2\pi} \int_{-\pi}^{\pi}
\lim_{n \To +\infty } g_n(\omega) e^{-j\omega k} d\omega$$

As $\abs{g_n(\omega)}<M$ we can appeal the dominated convergence
theorem to write
\begin{eqnarray*}
\frac{1}{2\pi} \int_{-\pi}^{\pi} \lim_{n \To +\infty } g_n(\omega)
e^{-j\omega k} d\omega & = & \lim_{n \To +\infty } \frac{1}{2\pi}
\int_{-\pi}^{\pi} g_n(\omega) e^{-j\omega k} d\omega \\ & = &
\frac{1}{2\pi} \sum_{k=-\infty}^{+\infty}
\int_{-\pi-2k\pi}^{\pi+2k\pi} X \left(\frac{\omega}{T}\right)
e^{-j\omega k} d\omega \\ & = & \frac{T}{2\pi} P.V.
\int_{-\infty}^{+\infty} X(\omega) e^{j\omega(-kT)} d\omega
\end{eqnarray*}

Now, as $x(t)$ is of bounded variation, the Jordan theorem on
Fourier transform inversion says that $$\frac{1}{2\pi} P.V.
\int_{-\infty}^{+\infty} X(\omega) e^{j\omega(-kT)} d\omega =
\frac{x(-kT^+)+x(-kT^-)}{2}$$ and the result follows.

\end{proof}

\noindent There are, however, very \PMlinkescapetext{simple functions} which do not
satisfy the conditions required in the theorem above. Such is the
case of the \emph{pulse} function $$rect(t)= \begin{cases}
1 & \text{if} \abs{t} < \frac{1}{2},\\
0 & \text{if} \abs{t} \geq \frac{1}{2}.
\end{cases}$$ whose Fourier transform is $\frac{sin(x/2)}{x/2}$ -with the definition made in
footnote 1-. $sinc(x)$ behaves as $\frac{1}{x}$, so
$\sum_{k=-\infty}^{+\infty} sinc \left(\frac{\omega- 2\pi \cdot
k}{T} \right)$ will not converge in general \footnote{for monotone
decreasing functions "series behaves as integrals", that is, if
$\sum_{k=0}^{+\infty} f(x-k \cdot M)$ converges or diverges so does
$\int_{0}^{+\infty} f$}; it will converge for those $T$ that make
the series an \PMlinkname{alternating}{AlternatingSeries} one, but not for the rest values of $T$.
Therefore we need another result which somehow relates both sides of
eq \ref{eq1}.

\bigskip

\noindent As we have pointed out, the problem with the pulse
function is that its Fourier transform does not decay rapid enough
for the series to converge. So we will try smoothing the signal out
so that its Fourier transform will decay faster and, hopefully, the
series converges. We wish the smoothed version of $x(t)$ to resemble
the original signal, so uniform approximation seems reasonable. But,
as we will see, for an infinite number of samples $\{nT, n \in \Z\}$
each of these might require a different degree of approximation and
it could be impossible to find an uniform approximation for all the
samples. So, we will focus on time limited signal, for which we have
the following result.

\begin{notat}
$\D(\R)$ will denote the set of test functions on $\R$ and $\rapid$
the set of rapidly decreasing functions on $\R$ -the Schwartz
space-. The symbol $\hat{}$ above a function will denote its Fourier
transform. We know that $\D(\R) \subset \rapid$ and that the Fourier
transform is an isomorphism of $\rapid$ onto itself. The product of
two functions $X$ and $\phi$ at $\omega$ will be denoted by
$X\phi(\omega)$. Convolution will be denoted by $\ast$.
\end{notat}

\medskip

\begin{thm}
Let $\phi \in \D(\R)$ be a test function and $\{\phi_j(t)=j \cdot
\phi(j \cdot t),\,\, j=1,2,\ldots \}$ be an approximate identity.
Let $x(t)$ be a time limited bounded variation signal. If
$\replica{X\hat{\phi_j}}{-\infty}{+\infty}$ converges for all
$j=1,2,\ldots$ to a continuous function, then \label{eq2}
\begin{equation}
\lim_{j \rightarrow \infty}\frac {1}{T}
\replica{X\hat{\phi_j}}{-\infty}{+\infty} = \DTFT{x}
\end{equation}
\end{thm}

\begin{proof}
Take the signal $(x \ast \phi_j)(t)$ whose Fourier transform is
$X\hat{\phi}(\omega)$. This signal satisfies, by hypothesis, the
conditions of Theorem 1, so we can write $$\frac {1}{T}
\replica{X\hat{\phi_j}}{-\infty}{+\infty} = \DTFT{(x \ast \phi_j)}
$$

The right hand side is actually a sum of a finite number of terms
since the signal $(x \ast \phi_j)(t)$ is time limited -being the
convolution of two compact supported functions-. This makes the
Fourier series a continuous function, which, together with the
hypothesis that $\frac {1}{T}
\replica{X\hat{\phi_j}}{-\infty}{+\infty}$ is continuous, shows that
equality in the above equation is pointwise.

\noindent Now let $j \rightarrow \infty$ and use the fact that
$\{\phi_j(t)\}$ is an approximate identity to obtain equation
(\ref{eq2}).
\end{proof}

\noindent The rationale behind choosing test functions in the last
theorem is that, in most
cases,$\replica{X\hat{\phi_j}}{-\infty}{+\infty}$ will converge even
though $\replica{X}{-\infty}{+\infty}$ do not. This is because
rapidly decreasing functions decay faster than $\frac{1}{x^\alpha}$
for any $\alpha$. So, for example, the Fourier transform of the
pulse function has been tamed enough to make the series converge.


\begin{rem}
When the signal $x(t)$ is continuous, the right hand side of eqs.
(\ref{eq1}) or (\ref{eq2}) reads $$ \sum_{k=-\infty}^{+\infty}
x_d[k] e^{-j\omega k}$$ where $x_d[k]=x(kT)$. This is defined as the
\emph{(Discrete Time) Fourier Transform, DTFT} of the sequence
${x_d[n]=x(nT)}$
\end{rem}

%%%%%
%%%%%
\end{document}
