\documentclass[12pt]{article}
\usepackage{pmmeta}
\pmcanonicalname{EllipticCurveDiscreteLogarithmProblem}
\pmcreated{2013-03-22 13:46:01}
\pmmodified{2013-03-22 13:46:01}
\pmowner{mathcam}{2727}
\pmmodifier{mathcam}{2727}
\pmtitle{elliptic curve discrete logarithm problem}
\pmrecord{7}{34471}
\pmprivacy{1}
\pmauthor{mathcam}{2727}
\pmtype{Definition}
\pmcomment{trigger rebuild}
\pmclassification{msc}{94A60}
\pmsynonym{elliptic curve discrete log problem}{EllipticCurveDiscreteLogarithmProblem}
\pmrelated{DiffieHellmanKeyExchange}
\pmrelated{ArithmeticOfEllipticCurves}

% this is the default PlanetMath preamble.  as your knowledge
% of TeX increases, you will probably want to edit this, but
% it should be fine as is for beginners.

% almost certainly you want these
\usepackage{amssymb}
\usepackage{amsmath}
\usepackage{amsfonts}

% used for TeXing text within eps files
%\usepackage{psfrag}
% need this for including graphics (\includegraphics)
%\usepackage{graphicx}
% for neatly defining theorems and propositions
%\usepackage{amsthm}
% making logically defined graphics
%%%\usepackage{xypic}

% there are many more packages, add them here as you need them

% define commands here
\begin{document}
The elliptic curve discrete logarithm problem is the cornerstone of much of present-day elliptic curve cryptography.  It relies on the natural group law on a non-singular elliptic curve which allows one to add points on the curve together.  Given an elliptic curve $E$ over a finite field $F$, a point on that curve, $P$, and another point you know to be an integer multiple of that point, $Q$, the ``problem'' is to find the integer $n$ such that $nP=Q$.

The problem is computationally difficult unless the curve has a ``bad'' number of points over the given field, where the term ``bad'' encompasses various collections of numbers of points which make the elliptic curve discrete logarithm problem breakable.  For example, if the number of points on $E$ over $F$ is the same as the number of elements of $F$, then the curve is vulnerable to attack.  It is because of these issues that point-counting on elliptic curves is such a hot topic in elliptic curve cryptography.

For an introduction to point-counting, reference Schoof's algorithm.
%%%%%
%%%%%
\end{document}
