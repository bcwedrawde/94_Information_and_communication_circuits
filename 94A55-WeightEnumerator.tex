\documentclass[12pt]{article}
\usepackage{pmmeta}
\pmcanonicalname{WeightEnumerator}
\pmcreated{2013-03-22 15:13:23}
\pmmodified{2013-03-22 15:13:23}
\pmowner{GrafZahl}{9234}
\pmmodifier{GrafZahl}{9234}
\pmtitle{weight enumerator}
\pmrecord{4}{36987}
\pmprivacy{1}
\pmauthor{GrafZahl}{9234}
\pmtype{Definition}
\pmcomment{trigger rebuild}
\pmclassification{msc}{94A55}
\pmclassification{msc}{94B05}
\pmsynonym{Hamming weight enumerator}{WeightEnumerator}
%\pmkeywords{code}
%\pmkeywords{linear code}
%\pmkeywords{Hamming}
\pmrelated{KleeneStar}
\pmrelated{LinearCode}
\pmdefines{complete weight enumerator}
\pmdefines{weight distribution}
\pmdefines{Hamming weight distribution}

\endmetadata

% this is the default PlanetMath preamble.  as your knowledge
% of TeX increases, you will probably want to edit this, but
% it should be fine as is for beginners.

% almost certainly you want these
\usepackage{amssymb}
\usepackage{amsmath}
\usepackage{amsfonts}

% used for TeXing text within eps files
%\usepackage{psfrag}
% need this for including graphics (\includegraphics)
%\usepackage{graphicx}
% for neatly defining theorems and propositions
%\usepackage{amsthm}
% making logically defined graphics
%%%\usepackage{xypic}

% there are many more packages, add them here as you need them

% define commands here
\newcommand{\Prod}{\prod\limits}
\newcommand{\Sum}{\sum\limits}
\newcommand{\mbb}{\mathbb}
\newcommand{\mbf}{\mathbf}
\newcommand{\mc}{\mathcal}

% Math Operators/functions
\DeclareMathOperator{\cwe}{cwe}
\DeclareMathOperator{\we}{we}
\DeclareMathOperator{\wt}{wt}
\begin{document}
Let $A$ be an alphabet and $C$ a finite subset of $A^*$.
Then the \emph{complete weight enumerator} of $C$, denoted by
$\cwe_C$, is the polynomial in $|A|$ indeterminates $X_a$ labeled by
the letters of $a\in A$ with integer coefficients defined by
\begin{equation*}
\cwe_C((X_a)_{a\in A}):=\Sum_{c\in C}\Prod_{a\in A}X_a^{\wt_a(c)},
\end{equation*}
where $\wt_a(c)$ is the $a$-weight of the string $c$.

If $A$ is an abelian group, one defines the \emph{Hamming weight
  enumerator} of $C$, denoted by $\we_C$, as a polynomial in only two
  indeterminates $X$ and $Y$:
\begin{equation*}
\we_C(X,Y):=\cwe_C((X_a)_{a\in A})\vert_{\begin{array}{l}\scriptstyle X_0=X\\\scriptstyle X_a=Y\text{ if
  }a\neq 0\end{array}},
\end{equation*}
that is one distinguishes only between zero and the non-zero letters
of the strings in $C$.

If $C$ is a code of block length $n$, then both $\cwe_C$ and $\we_C$
are \PMlinkid{homogeneous of degree}{6577} $n$. Therefore, one can set $Y=1$ in $\we_C$
in this case without losing information. The resulting polynomial can
be uniquely rewritten in the form
\begin{equation*}
\we_C(X,1)=\Sum_{i=0}^nA_iX^{n-i},
\end{equation*}
the sequence $A_0,\ldots A_n$ defining the \emph{Hamming weight
  distribution}. Analogously, one can define more general weight
  distributions by setting all but one indeterminate in
  $\cwe_C((X_a)_{a\in A})$ equal to one.

\subsubsection*{Examples}
\begin{itemize}
\item Let $C$ be the ternary (that is $A=\mbb{F}_3=\{0,1,2\}$) linear
  code of block length $4$ \PMlinkescapetext{spanned by} the vectors $(1,1,1,1)$, $(1,1,0,0)$
  and $(1,0,1,0)$. Then
\begin{equation*}
\cwe_C(X_0,X_1,X_2)=X_0^4+4X_0^2X_1^2+4X_0^2X_1X_2+4X_0^2X_2^2+4X_0X_1^2X_2+4X_0X_1X_2^2+X_1^4+4X_1^2X_2^2+X_2^4
\end{equation*}
and
\begin{equation*}
\we_C(X,Y)=X^4+12X^2Y^2+8XY^3+6Y^4
\end{equation*}
and the Hamming weight distribution is $1,0,12,8,6$.
\item The Hamming weight enumerator of the full binary code of length
  $n$, $\mbb{F}_2^n$, is simply given by
  $\we_{\mbb{F}_2^n}(X,Y)=(X+Y)^n$, and the Hamming weight
  distribution is the $n$-th \PMlinkescapetext{row} of Pascal's triangle.
\end{itemize}
%%%%%
%%%%%
\end{document}
