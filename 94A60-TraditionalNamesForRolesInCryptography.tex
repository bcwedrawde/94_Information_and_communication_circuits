\documentclass[12pt]{article}
\usepackage{pmmeta}
\pmcanonicalname{TraditionalNamesForRolesInCryptography}
\pmcreated{2013-03-22 15:13:41}
\pmmodified{2013-03-22 15:13:41}
\pmowner{lieven}{1075}
\pmmodifier{lieven}{1075}
\pmtitle{traditional names for roles in cryptography}
\pmrecord{6}{36995}
\pmprivacy{1}
\pmauthor{lieven}{1075}
\pmtype{Definition}
\pmcomment{trigger rebuild}
\pmclassification{msc}{94A60}

\endmetadata

% this is the default PlanetMath preamble.  as your knowledge
% of TeX increases, you will probably want to edit this, but
% it should be fine as is for beginners.

% almost certainly you want these
\usepackage{amssymb}
\usepackage{amsmath}
\usepackage{amsfonts}

% used for TeXing text within eps files
%\usepackage{psfrag}
% need this for including graphics (\includegraphics)
%\usepackage{graphicx}
% for neatly defining theorems and propositions
%\usepackage{amsthm}
% making logically defined graphics
%%%\usepackage{xypic}

% there are many more packages, add them here as you need them

% define commands here
\begin{document}
In the field of cryptography, protocols are described and analysed to allow a number of parties to achieve certain goals like communication, authentication, voting etc. Initially these protocol descriptions used single letter variables in the style ``Let A and B be parties trying to communicate with the help of a mutually trusted entity C.''. This gave rise to a lot of repetition so a sort of pseudo-standard has arisen that uses first names for some of the standard roles.
Common are Alice and Bob as the first two parties in a protocol. If more parties are needed, the following ones are Carol and Dave. Eve is a passive attacker, an eavesdropper. Mallory is an active attacker. Trent is a trusted third party like a  mutually known key server. In the field of prove carrying code, there's Peggy the prover and Victor the verifier. The list at wikipedia (http://en.wikipedia.org/wiki/Characters\_in\_cryptography) contains even more elements from this bestiary.
%%%%%
%%%%%
\end{document}
