\documentclass[12pt]{article}
\usepackage{pmmeta}
\pmcanonicalname{ConvolutionAssociativityOf}
\pmcreated{2013-03-22 16:56:36}
\pmmodified{2013-03-22 16:56:36}
\pmowner{mps}{409}
\pmmodifier{mps}{409}
\pmtitle{convolution, associativity of}
\pmrecord{5}{39211}
\pmprivacy{1}
\pmauthor{mps}{409}
\pmtype{Derivation}
\pmcomment{trigger rebuild}
\pmclassification{msc}{94A12}
\pmclassification{msc}{44A35}

% this is the default PlanetMath preamble.  as your knowledge
% of TeX increases, you will probably want to edit this, but
% it should be fine as is for beginners.

% almost certainly you want these
\usepackage{amssymb}
\usepackage{amsmath}
\usepackage{amsfonts}

% used for TeXing text within eps files
%\usepackage{psfrag}
% need this for including graphics (\includegraphics)
%\usepackage{graphicx}
% for neatly defining theorems and propositions
\usepackage{amsthm}
% making logically defined graphics
%%%\usepackage{xypic}

% there are many more packages, add them here as you need them

% define commands here
\newtheorem*{proposition*}{Proposition}
\begin{document}
\begin{proposition*}
Convolution is associative.
\end{proposition*}

\begin{proof}

Let $f$, $g$, and $h$ be measurable functions on the reals, and
suppose the convolutions $(f*g)*h$ and $f*(g*h)$ exist.  We must show
that $(f*g)*h = f*(g*h)$.  By the definition of convolution,
\begin{align*}
((f*g)*h)(u)
&= \int_{\mathbb{R}} (f*g)(x) h(u-x)\,dx \\
&= \int_{\mathbb{R}} \left[ \int_{\mathbb{R}} f(y) g(x-y)\,dy\right] h(u-x)\,dx \\
&= \int_{\mathbb{R}} \int_{\mathbb{R}} f(y) g(x-y) h(u-x)\,dy\,dx.
\end{align*}
By Fubini's theorem we can switch the order of integration.  Thus
\begin{align*}
((f*g)*h)(u)
&= \int_{\mathbb{R}} \int_{\mathbb{R}} f(y) g(x-y) h(u-x)\,dx\,dy \\
&= \int_{\mathbb{R}} f(y) \left[\int_{\mathbb{R}} g(x-y) h(u-x)\,dx\right]\,dy.
\end{align*}
Now let us look at the inner integral.  By translation invariance,
\begin{align*}
\int_{\mathbb{R}} g(x-y) h(u-x)\,dx 
&= \int_{\mathbb{R}} g((x+y)-y) h(u-(x+y))\,dx \\
&= \int_{\mathbb{R}} g(x) h((u-y)-x)\,dx \\
&= (g*h)(u-y).
\end{align*}
So we have shown that
\[
((f*g)*h)(u) = \int_{\mathbb{R}} f(y)(g*h)(u-y)\,dy,
\]
which by definition is $(f*(g*h))(u)$.  Hence convolution is associative.
\end{proof}

\PMlinkescapeword{inner}
\PMlinkescapeword{order}
%%%%%
%%%%%
\end{document}
