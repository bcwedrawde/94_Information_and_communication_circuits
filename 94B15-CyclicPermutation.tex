\documentclass[12pt]{article}
\usepackage{pmmeta}
\pmcanonicalname{CyclicPermutation}
\pmcreated{2013-03-22 17:33:54}
\pmmodified{2013-03-22 17:33:54}
\pmowner{CWoo}{3771}
\pmmodifier{CWoo}{3771}
\pmtitle{cyclic permutation}
\pmrecord{13}{39974}
\pmprivacy{1}
\pmauthor{CWoo}{3771}
\pmtype{Definition}
\pmcomment{trigger rebuild}
\pmclassification{msc}{94B15}
\pmclassification{msc}{20B99}
\pmclassification{msc}{03-00}
\pmclassification{msc}{05A05}
\pmclassification{msc}{11Z05}
\pmclassification{msc}{94A60}
\pmsynonym{Caesar cipher}{CyclicPermutation}
\pmrelated{CyclicCode}
\pmrelated{SubgroupsWithCoprimeOrders}
\pmdefines{Caesar shift cipher}
\pmdefines{cyclic conjugate}

\usepackage{amssymb,amscd}
\usepackage{amsmath}
\usepackage{amsfonts}
\usepackage{mathrsfs}

% used for TeXing text within eps files
%\usepackage{psfrag}
% need this for including graphics (\includegraphics)
%\usepackage{graphicx}
% for neatly defining theorems and propositions
\usepackage{amsthm}
% making logically defined graphics
%%\usepackage{xypic}
\usepackage{pst-plot}
\usepackage{psfrag}

% define commands here
\newtheorem{prop}{Proposition}
\newtheorem{thm}{Theorem}
\newtheorem{ex}{Example}
\newcommand{\real}{\mathbb{R}}
\newcommand{\pdiff}[2]{\frac{\partial #1}{\partial #2}}
\newcommand{\mpdiff}[3]{\frac{\partial^#1 #2}{\partial #3^#1}}
\begin{document}
Let $A=\lbrace a_0,a_1,\ldots, a_{n-1}\rbrace$ be a finite set indexed by $i=0,\ldots, n-1$.  A \emph{cyclic permutation} on $A$ is a permutation $\pi$ on $A$ such that, for some integer $k$, $$\pi(a_i)=a_{(i+k)\!\!\!\!\!\! \pmod n},$$
where $a\!\!\pmod b:= a - \lfloor a/b \rfloor b$, the remainder of $a$ when divided by $b$, and $\lfloor \cdot \rfloor$ is the floor function.

For example, if $A=\lbrace 1,2,\ldots,m\rbrace$ such that $a_i=i+1$.  Then a cyclic permutation $\pi$ on $A$ has the form
\begin{eqnarray*}
\pi(1) &=& r \\
\pi(2) &=& r+1 \\
&\vdots & \\
\pi(m-r+1) &=& m \\
\pi(m-r+2) &=& 1 \\
&\vdots & \\
\pi(m) &=& r-1.
\end{eqnarray*}

In the usual permutation notation, it looks like 
$$\pi=\left( \begin{array}{ccccccc}
1 & 2 & \cdots & m-r+1 & m-r+2 & \cdots & m \\
r & r+1 & \cdots & m & 1 & \cdots & r-1 \end{array} \right)$$

\textbf{Remark}.  For every finite set of cardinality $n$, there are $n$ cyclic permutations.  Each non-trivial cyclic permutation has order $n$.  Furthermore, if $n$ is a prime number, the set of cyclic permutations forms a cyclic group.

\subsubsection*{Cyclic permutations on words}

Given a word $w=a_1a_2\cdots a_n$ on a set $\Sigma$ (may or may not be finite), a \emph{cyclic conjugate} of $w$ is a word $v$ derived from $w$ based on a cyclic permutation.  In other words, $v=\pi(a_1)\pi(a_2)\cdots \pi(a_n)$ for some cyclic permutation $\pi$ on $\lbrace a_1,\ldots, a_n\rbrace$.  Equivalently, $v$ and $w$ are cyclic conjugates of one another iff $w=st$ and $v=ts$ for some words $s,t$.

For example, the cyclic conjugates of the word $ababa$ over $\lbrace a,b\rbrace$ are $$baba^2,\quad aba^2b,\quad ba^2ba,\quad a^2bab,\quad\mbox{and itself}.$$
Strictly speaking, $\pi$ is a cyclic permutation on the \emph{multiset} $A=\lbrace a_1,\ldots, a_n\rbrace$, which can be thought of as a cyclic permutation on the set $A'=\lbrace (1,a_1),\ldots, (n,a_n)\rbrace$.  Furthermore, $\pi$ can be extended to a function on $A^*$: for every word $w=a_{\phi(1)}\cdots a_{\phi(m)}$, $\pi(w):=\pi(a_{\phi(1)})\cdots \pi(a_{\phi(m)})$, where $\phi$ is a permutation on $A$.

Given any word $w=a_1a_2\cdots a_n$ on $\Sigma$, two cyclic permutations $\pi_1,\pi_2$ on $\lbrace a_1,\ldots, a_n\rbrace$ are said to be the \emph{same} if $\pi_1(w)=\pi_2(w)$.  For example, with the word $abab$, then the cyclic permutation $$\left( \begin{array}{cccc}
1 & 2 & 3 & 4 \\
3 & 4 & 1 & 2 \end{array} \right)$$ is the same as the identity permutation.  There is a one-to-one correspondence between the set of all cyclic conjugates of $w$ and the set of all \emph{distinct} cyclic permutations on $\lbrace a_0,a_1,\ldots, a_n\rbrace$.

\textbf{Remarks}.
\begin{itemize}
\item
In a group $G$, if two elements $u,v$ are cyclic conjugates of one another, then they are conjugates: for if $u=st$ and $v=ts$, then $v=t(st)t^{-1}=tut^{-1}$.
\item
Cyclic permutations were used as a ciphering scheme by Julius Caesar.  Given an alphabet with letters, say $a,b,c, \ldots,x,y,z$, messages in letters are encoded so that each letter is shifted by three places.  For example, the name 
\begin{center} ``Julius Caesar'' becomes ``Mxolxv Fdhvdu''.\end{center} A ciphering scheme based on cyclic permutations is therefore also known as a \emph{Caesar shift cipher}.
\end{itemize}
%%%%%
%%%%%
\end{document}
