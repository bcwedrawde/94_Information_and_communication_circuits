\documentclass[12pt]{article}
\usepackage{pmmeta}
\pmcanonicalname{SignRelationalComplex}
\pmcreated{2013-03-22 17:53:11}
\pmmodified{2013-03-22 17:53:11}
\pmowner{Jon Awbrey}{15246}
\pmmodifier{Jon Awbrey}{15246}
\pmtitle{sign relational complex}
\pmrecord{6}{40369}
\pmprivacy{1}
\pmauthor{Jon Awbrey}{15246}
\pmtype{Definition}
\pmcomment{trigger rebuild}
\pmclassification{msc}{94A05}
\pmclassification{msc}{03G15}
\pmclassification{msc}{03E20}
\pmclassification{msc}{03B42}
\pmclassification{msc}{03B10}
\pmclassification{msc}{94A15}
\pmrelated{RelationTheory}
\pmrelated{SignRelation}
\pmrelated{SimplicialComplex}
\pmrelated{TriadicRelation}

\endmetadata

% this is the default PlanetMath preamble.  as your knowledge
% of TeX increases, you will probably want to edit this, but
% it should be fine as is for beginners.

% almost certainly you want these
\usepackage{amssymb}
\usepackage{amsmath}
\usepackage{amsfonts}

% used for TeXing text within eps files
%\usepackage{psfrag}
% need this for including graphics (\includegraphics)
%\usepackage{graphicx}
% for neatly defining theorems and propositions
%\usepackage{amsthm}
% making logically defined graphics
%%%\usepackage{xypic}

% there are many more packages, add them here as you need them

% define commands here

\begin{document}
\PMlinkescapephrase{basis}
\PMlinkescapephrase{Basis}
\PMlinkescapephrase{component}
\PMlinkescapephrase{Component}
\PMlinkescapephrase{components}
\PMlinkescapephrase{Components}
\PMlinkescapephrase{nor}
\PMlinkescapephrase{Nor}
\PMlinkescapephrase{object}
\PMlinkescapephrase{Object}
\PMlinkescapephrase{objects}
\PMlinkescapephrase{Objects}
\PMlinkescapephrase{potential}
\PMlinkescapephrase{Potential}

A \textbf{sign relational complex} is a generalization of a sign relation that allows for empty components in the \textit{elementary sign relations}, or sign relational triples of the form $(\mathrm{object}, \mathrm{sign}, \mathrm{interpretant})$.

Generally speaking, when it comes to things that are being contemplated as ostensible or potential signs of other things, neither the existence nor the uniqueness of the elements appearing in the sign relation is guaranteed.  For example, the reference of a putative sign to its putative objects may actually achieve reference to zero, to one, or to many objects.  A proper treatment of this complication calls for the conception of something slightly more general than a sign relation proper, namely, a sign relational complex.  Expressed in practical terms, this allows for \textit{missing data} in the columns of the relational database table for the sign relation in question.  Typically one operates on the default assumption that all of the roles of elementary sign relations are actually filled, but remains wary enough of the possible exceptions to deal with them on an \textit{ad hoc} basis.

%%%%%
%%%%%
\end{document}
