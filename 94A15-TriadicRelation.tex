\documentclass[12pt]{article}
\usepackage{pmmeta}
\pmcanonicalname{TriadicRelation}
\pmcreated{2013-03-22 17:47:53}
\pmmodified{2013-03-22 17:47:53}
\pmowner{Jon Awbrey}{15246}
\pmmodifier{Jon Awbrey}{15246}
\pmtitle{triadic relation}
\pmrecord{30}{40260}
\pmprivacy{1}
\pmauthor{Jon Awbrey}{15246}
\pmtype{Topic}
\pmcomment{trigger rebuild}
\pmclassification{msc}{94A15}
\pmclassification{msc}{94A05}
\pmclassification{msc}{03G15}
\pmclassification{msc}{03E20}
\pmclassification{msc}{03B42}
\pmclassification{msc}{03B10}
\pmsynonym{ternary relation}{TriadicRelation}
\pmrelated{LogicalMatrix}
\pmrelated{RelationTheory}
\pmrelated{SignRelation}
\pmrelated{SignRelationalComplex}
\pmrelated{SemioticEquivalenceRelation}

% this is the default PlanetMath preamble.  as your knowledge
% of TeX increases, you will probably want to edit this, but
% it should be fine as is for beginners.

% almost certainly you want these
\usepackage{amssymb}
\usepackage{amsmath}
\usepackage{amsfonts}

% used for TeXing text within eps files
%\usepackage{psfrag}
% need this for including graphics (\includegraphics)
%\usepackage{graphicx}
% for neatly defining theorems and propositions
%\usepackage{amsthm}
% making logically defined graphics
%%%\usepackage{xypic}

% there are many more packages, add them here as you need them

% define commands here

\begin{document}
In logic, mathematics, and semiotics, a \textbf{triadic relation} is an important special case of a polyadic or finitary relation, one in which the number of places in the relation is three.  In other language that is often used, a triadic relation is called a \textbf{ternary relation}.  One may also see the adjectives \textit{3-adic}, \textit{3-ary}, \textit{3-dimensional}, or \textit{3-place} being used to describe these relations.

Mathematics is positively rife with examples of 3-adic relations, and the concept of a \textit{sign relation}, a special case of a 3-adic relation, is fundamental to the field of semiotics, or the theory of signs.  Therefore it will be useful to consider a few concrete examples from each of these two realms.

\section{Examples from mathematics}

For the sake of topics to be taken up later it is useful to examine a pair of 3-adic relations in tandem, $L_0$ and $L_1$, that can be described in the following manner.

The first order of business is to define the space in which the relations $L_0$ and $L_1$ take up residence.  This space is constructed as a 3-fold cartesian power in the following way.

\begin{itemize}
\item
The \textit{boolean domain} is the set $\mathbb{B} = \{ 0, 1 \}.$
\item
The plus sign ``$+$'' denotes addition modulo 2.
\item
The third cartesian power of $\mathbb{B}$ is defined as follows:

\[ \mathbb{B}^3 = \mathbb{B} \times \mathbb{B} \times \mathbb{B} = \{ (x_1, x_2, x_3) : x_j \in \mathbb{B}\ \mbox{for}\ j = 1, 2, 3 \}. \]
\end{itemize}

In what follows, the space $X \times Y \times Z$ is isomorphic to $\mathbb{B} \times \mathbb{B} \times \mathbb{B} = \mathbb{B}^3.$

The relation $L_0$ is defined as follows:

\[ L_0 = \{ (x, y, z) \in \mathbb{B}^3 : x + y + z = 0 \}. \]

The relation $L_0$ is the set of four triples enumerated here:

\[ L_0 = \{ (0, 0, 0), (0, 1, 1), (1, 0, 1), (1, 1, 0) \}. \]

The relation $L_1$ is defined as follows:

\[ L_1 = \{ (x, y, z) \in \mathbb{B}^3 : x + y + z = 1 \}. \]

The relation $L_1$ is the set of four triples enumerated here:

\[ L_1 = \{ (0, 0, 1), (0, 1, 0), (1, 0, 0), (1, 1, 1) \}. \]

The triples that make up the relations $L_0$ and $L_1$ are conveniently arranged in the form of relational data tables, as follows:

\begin{quote}
\begin{tabular}{|c|c|c|}
\multicolumn{3}{c}{$L_0 = \{ (x, y, z) \in \mathbb{B}^3 : x + y + z = 0 \}$} \\
\hline\
$X$ & $Y$ & $Z$ \\
\hline\hline
 0  &  0  &  0  \\
\hline
 0  &  1  &  1  \\
\hline
 1  &  0  &  1  \\
\hline
 1  &  1  &  0  \\
\hline
\end{tabular}
\end{quote}

\begin{quote}
\begin{tabular}{|c|c|c|}
\multicolumn{3}{c}{$L_1 = \{ (x, y, z) \in \mathbb{B}^3 : x + y + z = 1 \}$} \\
\hline
$X$ & $Y$ & $Z$ \\
\hline\hline
 0  &  0  &  1  \\
\hline
 0  &  1  &  0  \\
\hline
 1  &  0  &  0  \\
\hline
 1  &  1  &  1  \\
\hline
\end{tabular}
\end{quote}

\section{Examples from semiotics}

The study of signs --- the full variety of significant forms of expression --- in relation to the things that signs are significant \textit{of}, and in relation to the beings that signs are significant \textit{to}, is known as \textit{semiotics} or the \textit{theory of signs}.  As just described, semiotics treats of a 3-place relation among \textit{signs}, their \textit{objects}, and their \textit{interpreters}.

The term \textit{semiosis} refers to any activity or process that involves signs.  Studies of semiosis that deal with its more abstract form are not concerned with every concrete detail of the entities that act as signs, as objects, or as agents of semiosis, but only with the most salient patterns of relationship among these three roles.  In particular, the formal theory of signs does not consider all of the properties of the interpretive agent but only the more striking features of the impressions that signs make on a representative interpreter.  In its formal aspects that impact or influence may be treated as just another sign, called the \textit{interpretant sign}, or the \textit{interpretant} for short.  A 3-adic relation among objects, signs, and interpretants is called a \textit{sign relation}.

For example, consider the aspects of sign use that concern two people, say, $\mathrm{Ann}$ and $\mathrm{Bob}$, in using their own proper names, $\mathrm{``Ann"}$ and $\mathrm{``Bob"}$, and in using the pronouns, $\mathrm{``I"}$ and $\mathrm{``you"}$.  For simplicity, these four signs may be abbreviated to form the set $\{ \mathrm{``A"}, \mathrm{``B"}, \mathrm{``i"}, \mathrm{``u"} \}$.  The abstract consideration of how $\mathrm{A}$ and $\mathrm{B}$ use this set of signs to refer to themselves and to each other leads to the contemplation of a pair of 3-adic relations, the sign relations $L_{\mathrm{A}}$ and $L_{\mathrm{B}}$, that reflect the differential use of these signs by $\mathrm{A}$ and $\mathrm{B}$, respectively.

Each of the sign relations, $L_{\mathrm{A}}$ and $L_{\mathrm{B}}$, consists of eight triples of the form $(x, y, z)$, where the object $x$ belongs to the object domain $O = \{ \mathrm{A}, \mathrm{B} \}$, where the sign $y$ belongs to the sign domain $S$, where the interpretant sign $z$ belongs to the interpretant domain $I$, and where it happens in this case that $S = I = \{ \mathrm{``A"}, \mathrm{``B"}, \mathrm{``i"}, \mathrm{``u"} \}$.  In general, it is convenient to refer to the union $S \cup I$ as the \textit{syntactic domain}, but in this case $S = I = S \cup I$.

The set-up so far can be summarized as follows:

\begin{itemize}
\item
$L_{\mathrm{A}}, L_{\mathrm{B}} \subseteq O \times S \times I.$
\item
$O = \{ \mathrm{A}, \mathrm{B} \}.$
\item
$S = \{ \mathrm{``A"}, \mathrm{``B"}, \mathrm{``i"}, \mathrm{``u"} \}.$
\item
$I = \{ \mathrm{``A"}, \mathrm{``B"}, \mathrm{``i"}, \mathrm{``u"} \}.$
\end{itemize}

The relation $L_{\mathrm{A}}$ is the set of eight triples enumerated here:

\[ \begin{array}{lccccr}
\{ & \mathrm{(A,``A",``A")}, & \mathrm{(A,``A",``i")}, & \mathrm{(A,``i",``A")}, & \mathrm{(A,``i",``i")}, & \\
   & \mathrm{(B,``B",``B")}, & \mathrm{(B,``B",``u")}, & \mathrm{(B,``u",``B")}, & \mathrm{(B,``u",``u")}  & \}.
\end{array} \]

The triples in $L_{\mathrm{A}}$ represent the way that interpreter $\mathrm{A}$ uses signs.  For example, the listing of the triple $\mathrm{(B,``u",``B")}$ in $L_{\mathrm{A}}$ represents the fact that $\mathrm{A}$ uses $\mathrm{``B"}$ to mean the same thing that $\mathrm{A}$ uses $\mathrm{``u"}$ to mean, namely, $\mathrm{B}$.

The relation $L_{\mathrm{B}}$ is the set of eight triples enumerated here:

\[ \begin{array}{lccccr}
\{ & \mathrm{(A,``A",``A")}, & \mathrm{(A,``A",``u")}, & \mathrm{(A,``u",``A")}, & \mathrm{(A,``u",``u")}, & \\
   & \mathrm{(B,``B",``B")}, & \mathrm{(B,``B",``i")}, & \mathrm{(B,``i",``B")}, & \mathrm{(B,``i",``i")}  & \}.
\end{array} \]

The triples in $L_{\mathrm{B}}$ represent the way that interpreter $\mathrm{B}$ uses signs. For example, the listing of the triple $\mathrm{(B,``i",``B")}$ in $L_{\mathrm{B}}$ represents the fact that $\mathrm{B}$ uses $\mathrm{``B"}$ to mean the same thing that $\mathrm{B}$ uses $\mathrm{``i"}$ to mean, namely, $\mathrm{B}$.

The triples that make up the relations $L_{\mathrm{A}}$ and $L_{\mathrm{B}}$ are conveniently arranged in the form of relational data tables, as follows:

\begin{quote}
\begin{tabular}{|c|c|c|}
\multicolumn{3}{c}{$L_{\mathrm{A}}$ = Sign Relation of Interpreter A} \\
\hline\
Object       & Sign            & Interpretant     \\
\hline\hline
$\mathrm{A}$ & $\mathrm{``A"}$ & $\mathrm{``A"}$  \\
\hline
$\mathrm{A}$ & $\mathrm{``A"}$ & $\mathrm{``i"}$  \\
\hline
$\mathrm{A}$ & $\mathrm{``i"}$ & $\mathrm{``A"}$  \\
\hline
$\mathrm{A}$ & $\mathrm{``i"}$ & $\mathrm{``i"}$  \\
\hline
$\mathrm{B}$ & $\mathrm{``B"}$ & $\mathrm{``B"}$  \\
\hline
$\mathrm{B}$ & $\mathrm{``B"}$ & $\mathrm{``u"}$  \\
\hline
$\mathrm{B}$ & $\mathrm{``u"}$ & $\mathrm{``B"}$  \\
\hline
$\mathrm{B}$ & $\mathrm{``u"}$ & $\mathrm{``u"}$  \\
\hline
\end{tabular}
\end{quote}

\begin{quote}
\begin{tabular}{|c|c|c|}
\multicolumn{3}{c}{$L_{\mathrm{B}}$ = Sign Relation of Interpreter B} \\
\hline\
Object       & Sign            & Interpretant     \\
\hline\hline
$\mathrm{A}$ & $\mathrm{``A"}$ & $\mathrm{``A"}$  \\
\hline
$\mathrm{A}$ & $\mathrm{``A"}$ & $\mathrm{``u"}$  \\
\hline
$\mathrm{A}$ & $\mathrm{``u"}$ & $\mathrm{``A"}$  \\
\hline
$\mathrm{A}$ & $\mathrm{``u"}$ & $\mathrm{``u"}$  \\
\hline
$\mathrm{B}$ & $\mathrm{``B"}$ & $\mathrm{``B"}$  \\
\hline
$\mathrm{B}$ & $\mathrm{``B"}$ & $\mathrm{``i"}$  \\
\hline
$\mathrm{B}$ & $\mathrm{``i"}$ & $\mathrm{``B"}$  \\
\hline
$\mathrm{B}$ & $\mathrm{``i"}$ & $\mathrm{``i"}$  \\
\hline
\end{tabular}
\end{quote}

%%%%%
%%%%%
\end{document}
