\documentclass[12pt]{article}
\usepackage{pmmeta}
\pmcanonicalname{ProofOfGaussianMaximizesEntropyForGivenCovariance}
\pmcreated{2013-03-22 12:19:35}
\pmmodified{2013-03-22 12:19:35}
\pmowner{Mathprof}{13753}
\pmmodifier{Mathprof}{13753}
\pmtitle{proof of Gaussian maximizes entropy for given covariance}
\pmrecord{10}{31948}
\pmprivacy{1}
\pmauthor{Mathprof}{13753}
\pmtype{Proof}
\pmcomment{trigger rebuild}
\pmclassification{msc}{94A17}
\pmrelated{QuadraticForm}
\pmrelated{RelativeEntropy}
\pmrelated{MultidimensionalGaussianIntegral}

\endmetadata

% this is the default PlanetMath preamble.  as your knowledge
% of TeX increases, you will probably want to edit this, but
% it should be fine as is for beginners.

% almost certainly you want these
\usepackage{amssymb}
\usepackage{amsmath}
\usepackage{amsfonts}

% used for TeXing text within eps files
%\usepackage{psfrag}
% need this for including graphics (\includegraphics)
%\usepackage{graphicx}
% for neatly defining theorems and propositions
%\usepackage{amsthm}
% making logically defined graphics
%%%\usepackage{xypic} 

% there are many more packages, add them here as you need them

% define commands here
\newcommand{\mv}[1]{\mathbf{#1}}	% matrix or vector
\newcommand{\cov}{\mathrm{cov}}
\newcommand{\mvt}[1]{\mv{#1}^{\mathrm{T}}}
\newcommand{\mvi}[1]{\mv{#1}^{-1}}
\newcommand{\mpderiv}[1]{\frac{\partial}{\partial {#1}}}
\begin{document}
Let $f, K, \phi$ be as in the \PMlinkname{parent}{GaussianMaximizesEntropyForGivenCovariance} entry.

The proof uses the nonnegativity of relative entropy $D(f||\phi)$, and an interesting  property of quadratic forms.  If $A$ is a quadratic form and $p,q$ are probability distributions each with mean $\mv{0}$ and covariance matrix $\mv{K}$, we have

\begin{equation}
\int p\ x_i x_j\ dx_i dx_j = K_{ij} = \int q\ x_i x_j\ dx_i dx_j
\end{equation}
and thus
\begin{equation}
\int A p = \int A q
\end{equation}

Now note that since
\begin{equation}
\phi(\mv{x}) = \left((2\pi)^n |\mv{K}| \right)^{-\frac{1}{2}} \exp{(- \frac{1}{2} \mvt{x} \mv{K}^{-1} \mv{x})},
\end{equation}
we see that $\log \phi$ is a quadratic form plus a constant.

\begin{eqnarray*}
\textbf{0} &\le D(f||\phi)\\
&= \int f \log \frac{f}{\phi}\\
&= \int f \log f - \int f \log \phi\\
&= -h(f) - \int f \log \phi\\
&= -h(f) - \int \phi \log \phi \qquad \text{by the quadratic form property above}\\
&= -h(f) + h(\phi)
\end{eqnarray*}
and thus $h(\phi) \ge h(f)$.
%%%%%
%%%%%
\end{document}
