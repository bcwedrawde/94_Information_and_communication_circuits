\documentclass[12pt]{article}
\usepackage{pmmeta}
\pmcanonicalname{DiffieHellmanKeyExchange}
\pmcreated{2013-03-22 13:45:58}
\pmmodified{2013-03-22 13:45:58}
\pmowner{mathcam}{2727}
\pmmodifier{mathcam}{2727}
\pmtitle{Diffie-Hellman key exchange}
\pmrecord{6}{34470}
\pmprivacy{1}
\pmauthor{mathcam}{2727}
\pmtype{Algorithm}
\pmcomment{trigger rebuild}
\pmclassification{msc}{94A60}
\pmrelated{EllipticCurveDiscreteLogarithmProblem}
\pmrelated{ArithmeticOfEllipticCurves}

% this is the default PlanetMath preamble.  as your knowledge
% of TeX increases, you will probably want to edit this, but
% it should be fine as is for beginners.

% almost certainly you want these
\usepackage{amssymb}
\usepackage{amsmath}
\usepackage{amsfonts}

% used for TeXing text within eps files
%\usepackage{psfrag}
% need this for including graphics (\includegraphics)
%\usepackage{graphicx}
% for neatly defining theorems and propositions
%\usepackage{amsthm}
% making logically defined graphics
%%%\usepackage{xypic}

% there are many more packages, add them here as you need them

% define commands here
\begin{document}
The Diffie-Hellman key exchange is a cryptographic protocol for symmetric key exchange.  There are various implementations of this protocol.  The following interchange between Alice and Bob demonstrates the Elliptic Curve Diffie-Hellman key exchange.
\begin{itemize}
\item 1)  Alice and Bob publicly agree on an elliptic curve $E$ over a large finite field $F$ and a point $P$ on that curve.
\item 2)  Alice and Bob each privately choose large random integers, denoted $a$ and $b$.
\item 3)  Using elliptic curve point-addition, Alice computes $aP$ on $E$ and sends it to Bob.  Bob computes $bP$ on $E$ and sends it to Alice.
\item 4)  Both Alice and Bob can now compute the point $abP$, Alice by multipliying the received value of $bP$ by her secret number $a$, and Bob vice-versa.
\item 5)  Alice and Bob agree that the $x$ coordinate of this point will be their shared secret value.
\end{itemize}

An evil interloper Eve observing the communications will be able to intercept only the objects $E$, $P$, $aP$, and $bP$.  She can succeed in determining the final secret value by gaining knowledge of either of the values $a$ or $b$.  Thus, the security of the exchange depends on the hardness of that problem, known as the elliptic curve discrete logarithm problem.  For large $a$ and $b$, it is a computationally ``difficult'' problem.

As a side note, some care has to be taken to choose an appropriate curve $E$.  Singular curves and ones with ``bad" numbers of points on it (over the given field) have simplified solutions to the discrete log problem.
%%%%%
%%%%%
\end{document}
