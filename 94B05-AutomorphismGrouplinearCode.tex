\documentclass[12pt]{article}
\usepackage{pmmeta}
\pmcanonicalname{AutomorphismGrouplinearCode}
\pmcreated{2013-03-22 15:18:40}
\pmmodified{2013-03-22 15:18:40}
\pmowner{GrafZahl}{9234}
\pmmodifier{GrafZahl}{9234}
\pmtitle{automorphism group (linear code)}
\pmrecord{5}{37114}
\pmprivacy{1}
\pmauthor{GrafZahl}{9234}
\pmtype{Definition}
\pmcomment{trigger rebuild}
\pmclassification{msc}{94B05}
\pmsynonym{automorphism group}{AutomorphismGrouplinearCode}
%\pmkeywords{code}
%\pmkeywords{transform}
\pmrelated{LinearCode}
\pmdefines{monomial transform}
\pmdefines{equivalent}
\pmdefines{equivalent code}
\pmdefines{automorphism}
\pmdefines{permutation group}

\endmetadata

% this is the default PlanetMath preamble.  as your knowledge
% of TeX increases, you will probably want to edit this, but
% it should be fine as is for beginners.

% almost certainly you want these
\usepackage{amssymb}
\usepackage{amsmath}
\usepackage{amsfonts}
\usepackage[latin1]{inputenc}

% used for TeXing text within eps files
%\usepackage{psfrag}
% need this for including graphics (\includegraphics)
%\usepackage{graphicx}
% for neatly defining theorems and propositions
\usepackage{amsthm}
% making logically defined graphics
%%%\usepackage{xypic}

% there are many more packages, add them here as you need them

% define commands here
\newcommand{\Bigcup}{\bigcup\limits}
\newcommand{\Prod}{\prod\limits}
\newcommand{\Sum}{\sum\limits}
\newcommand{\mbb}{\mathbb}
\newcommand{\mbf}{\mathbf}
\newcommand{\mc}{\mathcal}
\newcommand{\mmm}[9]{\left(\begin{array}{rrr}#1&#2&#3\\#4&#5&#6\\#7&#8&#9\end{array}\right)}
\newcommand{\mf}{\mathfrak}
\newcommand{\ol}{\overline}

% Math Operators/functions
\DeclareMathOperator{\Aut}{Aut}
\DeclareMathOperator{\Frob}{Frob}
\DeclareMathOperator{\cwe}{cwe}
\DeclareMathOperator{\id}{id}
\DeclareMathOperator{\we}{we}
\DeclareMathOperator{\wt}{wt}
\begin{document}
\PMlinkescapeword{difference}
Let $\mbb{F}_q$ be the finite field with $q$ elements. The group
$\mc{M}_{n,q}$ of $n\times n$ monomial matrices with entries in $\mbb{F}_q$
acts on the set $\mf{C}_{n,q}$ of linear codes over $\mbb{F}_q$ of
block length $n$ via the \emph{monomial transform}: let $M=(M_{ij})_{i,j=1}^n\in\mc{M}_{n,q}$ and $C\in\mf{C}_{n,q}$ and set
\begin{equation*}
C_M:=\left\{\left(\Sum_{i=1}^nM_{i1}c_i,\ldots,\Sum_{i=1}^nM_{in}c_i\right)\mid(c_1,\ldots,c_n)\in C\right\}.
\end{equation*}
This definition looks quite complicated, but since $M$ is \PMlinkescapetext{monomial}, it
really just means that $C_M$ is the linear code obtained from $C$ by
permuting its coordinates and then multiplying each coordinate with
some nonzero element from $\mbb{F}_q$.

Two linear codes lying in the same orbit with respect to this action
are said to be \emph{equivalent}. The isotropy subgroup of $C$
is its \emph{automorphism group}, denoted by $\Aut(C)$. The elements
of $\Aut(C)$ are the \emph{automorphisms} of $C$.

Sometimes one is only interested in the action of the permutation
matrices on $\mf{C}_{n,q}$. The permutation matrices form a subgroup
of $\mc{M}_{n,q}$ and the resulting subgroup of the automorphism group
$\Aut(C)$ of a linear code $C\in\mf{C}_{n,q}$ is called the
\emph{permutation group}. In the case of binary codes, this doesn't
make any difference, since the finite field $\mbb{F}_2$ contains only
one nonzero element.
%%%%%
%%%%%
\end{document}
