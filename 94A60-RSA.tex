\documentclass[12pt]{article}
\usepackage{pmmeta}
\pmcanonicalname{RSA}
\pmcreated{2013-03-22 15:14:50}
\pmmodified{2013-03-22 15:14:50}
\pmowner{aoh45}{5079}
\pmmodifier{aoh45}{5079}
\pmtitle{RSA}
\pmrecord{5}{37025}
\pmprivacy{1}
\pmauthor{aoh45}{5079}
\pmtype{Definition}
\pmcomment{trigger rebuild}
\pmclassification{msc}{94A60}
\pmsynonym{RSA cryptosystem}{RSA}

\endmetadata

\usepackage{amssymb}
\usepackage{amsmath}
\usepackage{amsfonts}
\usepackage{amsthm}
\begin{document}
\emph{RSA} is an example of public key cryptography. It is a widely used system, relying for its security on the difficulty of factoring a large number.

Alice forms her public and private keys as follows:
\begin{itemize}
\item Chooses large primes $p$ and $q$, then form $n=pq$.
\item Chooses e coprime with $\phi (n) = (p-1)(q-1)$.
\item Publishes $(n,e)$ as her public key.
\item Computes private key $d$ such that $de \equiv 1 \quad\textrm{mod } \phi(n)$.
\end{itemize}

To encrypt a message $M$ (where $M < n$) the user Bob  forms $C = M^e \quad\textrm{mod } n$.

To decrypt the message, Alice forms $d(C) = C^d \quad\textrm{mod } n$.

This recovers message $M$ because:
\begin{align*}
d(C) &= C^d \quad\textrm{mod } n \\
     &= (M^e + rn)^d \quad\textrm{mod } n \qquad \textrm{for some }r \\
     &= (M^{(p-1)})^{t(q-1)}M \quad\textrm{mod } n \qquad \textrm{for some }t\\
     &\equiv (1+sp)^{t(q-1)}M \quad\textrm{mod } p \qquad \textrm{for some }s \\
     &\equiv M \quad\textrm{mod } p
\end{align*}

So $d(C) \equiv M \quad\textrm{mod } p$ and similarly, $d(C) \equiv M \quad\textrm{mod } q$ so by the Chinese remainder theorem, $d(C) \equiv M \quad\textrm{mod } n$, and since we know $M<n$ we know that $M = d(C)$.
%%%%%
%%%%%
\end{document}
