\documentclass[12pt]{article}
\usepackage{pmmeta}
\pmcanonicalname{HammingDistance}
\pmcreated{2013-03-22 12:07:05}
\pmmodified{2013-03-22 12:07:05}
\pmowner{Mathprof}{13753}
\pmmodifier{Mathprof}{13753}
\pmtitle{Hamming distance}
\pmrecord{11}{31263}
\pmprivacy{1}
\pmauthor{Mathprof}{13753}
\pmtype{Definition}
\pmcomment{trigger rebuild}
\pmclassification{msc}{94C99}
\pmclassification{msc}{05C12}
\pmrelated{Metric}
\pmrelated{HammingMetric}

\endmetadata


\begin{document}
\PMlinkescapeword{information}
\PMlinkescapeword{length}
\PMlinkescapeword{simple}

\section{Hamming Distance}

In comparing two bit patterns, the \emph{Hamming distance} is the count of bits different in the two patterns. More generally, if two ordered lists of items are compared, the Hamming distance is the number of items that do not identically agree. This distance is applicable to encoded information, and is a particularly simple metric of comparison, often more useful than the city-block distance or Euclidean distance. 

The Hamming distance is a true metric, as it induces a metric space on the set of ordered lists of some fixed length.

\begin{thebibliography}{3}

\bibitem{DAB} The Data Analysis Briefbook, \PMlinkexternal{http://rkb.home.cern.ch/rkb/titleA.html}{http://rkb.home.cern.ch/rkb/titleA.html}

\end{thebibliography}
%%%%%
%%%%%
%%%%%
\end{document}
