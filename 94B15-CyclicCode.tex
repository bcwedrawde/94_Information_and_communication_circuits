\documentclass[12pt]{article}
\usepackage{pmmeta}
\pmcanonicalname{CyclicCode}
\pmcreated{2013-03-22 15:12:56}
\pmmodified{2013-03-22 15:12:56}
\pmowner{GrafZahl}{9234}
\pmmodifier{GrafZahl}{9234}
\pmtitle{cyclic code}
\pmrecord{6}{36978}
\pmprivacy{1}
\pmauthor{GrafZahl}{9234}
\pmtype{Definition}
\pmcomment{trigger rebuild}
\pmclassification{msc}{94B15}
\pmrelated{LinearCode}
\pmrelated{Code}

\endmetadata

% this is the default PlanetMath preamble.  as your knowledge
% of TeX increases, you will probably want to edit this, but
% it should be fine as is for beginners.

% almost certainly you want these
\usepackage{amssymb}
\usepackage{amsmath}
\usepackage{amsfonts}

% used for TeXing text within eps files
%\usepackage{psfrag}
% need this for including graphics (\includegraphics)
%\usepackage{graphicx}
% for neatly defining theorems and propositions
%\usepackage{amsthm}
% making logically defined graphics
%%%\usepackage{xypic}

% there are many more packages, add them here as you need them

% define commands here
\begin{document}
Let $C$ be a linear code over a finite field $A$ of block length $n$. $C$ is called a \emph{cyclic code}, if for every codeword $c=(c_1,\ldots,c_n)$ from $C$, the word $(c_n,c_1,\ldots,c_{n-1})\in A^n$ obtained by a \PMlinkescapetext{cyclic} right shift of \PMlinkescapetext{components} is also a codeword from $C$.

Sometimes, $C$ is called \emph{the} $c$-cyclic code, if $C$ is the smallest cyclic code containing $c$, or, in other words, $C$ is the linear code generated by $c$ and all codewords obtained by \PMlinkescapetext{cyclic} shifts of its \PMlinkescapetext{components}.

For example, if $A=\mathbb{F}_2$ and $n=3$, the codewords contained in the $(1,1,0)$-cyclic code are precisely
$$(0,0,0), (1,1,0), (0,1,1)\text{ and }(1,0,1).$$

Trivial examples of cyclic codes are $A^n$ itself and the code containing only the zero codeword.
%%%%%
%%%%%
\end{document}
