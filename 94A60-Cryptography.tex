\documentclass[12pt]{article}
\usepackage{pmmeta}
\pmcanonicalname{Cryptography}
\pmcreated{2013-03-22 15:49:07}
\pmmodified{2013-03-22 15:49:07}
\pmowner{CWoo}{3771}
\pmmodifier{CWoo}{3771}
\pmtitle{cryptography}
\pmrecord{10}{37784}
\pmprivacy{1}
\pmauthor{CWoo}{3771}
\pmtype{Definition}
\pmcomment{trigger rebuild}
\pmclassification{msc}{94A60}
\pmsynonym{cleartext}{Cryptography}
\pmsynonym{cyphertext}{Cryptography}
\pmsynonym{cryptotext}{Cryptography}
\pmdefines{plaintext}
\pmdefines{ciphertext}
\pmdefines{message space}
\pmdefines{cyphertext space}
\pmdefines{key space}
\pmdefines{key}
\pmdefines{key pair}
\pmdefines{encryption function}
\pmdefines{decryption function}
\pmdefines{cryptosystem}
\pmdefines{cryptanalysis}

% this is the default PlanetMath preamble.  as your knowledge
% of TeX increases, you will probably want to edit this, but
% it should be fine as is for beginners.

% almost certainly you want these
\usepackage{amssymb}
\usepackage{amsmath}
\usepackage{amsfonts}

% used for TeXing text within eps files
%\usepackage{psfrag}
% need this for including graphics (\includegraphics)
%\usepackage{graphicx}
% for neatly defining theorems and propositions
%\usepackage{amsthm}
% making logically defined graphics
%%%\usepackage{xypic}

% there are many more packages, add them here as you need them

% define commands here
\begin{document}
Cryptography is the science of encoding and decoding messages so that they are only understood by intended senders and recipients.

Informally, the basic ingredients in cryptography consist of 
\begin{enumerate}
\item An alphabet $\Sigma$.  A word over $\Sigma$ is called a \emph{plaintext} or \emph{cleartext}.  The set $M:=\Sigma^*$ of all plaintexts is called the \emph{message space}.
\item An alphabet $\Delta$.  A word over $\Delta$ is called a \emph{ciphertext} or \emph{cryptotext}.  The set $C:=\Delta^*$ of all ciphertexts is called the \emph{ciphertext space}.
\item A set $K$ called the key space, such that each $k\in K$, called a \emph{key}, determines functions $e_k: M\to C$ and $d_k:C\to M$ called \emph{encryption function} and \emph{decryption function} respectively.  It is generally required that $d_k\circ e_k = 1_M$ (and usually $e_k\circ d_k=1_C$ so that $e_k$ and $d_k$ are inverses of one another).  The pair $(e_k,d_k)$ is usually called a \emph{key pair}.
\end{enumerate}
The triple $(M,C,K)$ is called a \emph{cryptosystem}.

Given a cryptosystem, one generally wants both $e_k$ and $d_k$ to be easily computable, so that encryption by the sender and decryption by the intended receiver can be done effortlessly.  For example, if $A$ were to send $B$ an encrypted message, $A$ needs to be able to \emph{easily} encrypt a plaintext $u$ into a ciphertext $v=e_k(u)$.  Upon receiving $v$, the intended receiver $B$ needs to be able to \emph{easily} decrypt $v$ into plaintext $u=d_k(v)$.  On the other hand, one also wants the task of recovering the plaintext from the ciphertext very difficult, or intractible, without knowing the decryption functions $d_k$.  For example, if an eavesdropper $C$ ever gets hold of $v$, he/she will have a very hard time recovering $u$ without knowing $d_k$, sometimes even in the presence of knowing $e_k$.

Here, the adjectives ``easy'', ``difficult'', ``intractible'' are measured in terms of the complexity involved in the computations.  For example, ``easy'' could mean that the time involved in the computations depends linearly on the length of the input $u$ (hopefully with a small coefficient), whereas ``difficult'' could mean the dependence be exponential instead.  A function $f$ such that it is easy to compute $f(u)$ given $u$ but very hard to find $u$ given $f(u)$ is usually called a \emph{one-way function}.  For example, $f(m,n)=mn$ where $m,n$ are primes, is one-way, or nearly so.  It is easy to multiply, but very hard to factor, particularly when both $m$ and $n$ are large.

The study of cryptogrpahy thus can be loosely broken up into two main branches: the construction of a good cryptosystem (meaning that the encryption functions should be one-way, and the decryption functions should be easy to compute), and the breaking of a ciphertext in some given setting (for example, where the encryption functions are known).  The latter of the two branches is also known as \emph{cryptanalysis}.

The mathematics behind cryptography involves a variety of disciplines, of which the main ones are information theory, formal languages, computational complexity, probability theory, and number theory.

Cryptographic methods range from simple additive ciphers to sophisticated public key encryption systems, first introduced by Diffie and Hellman in the mid 1970's.
%%%%%
%%%%%
\end{document}
